\documentclass{article}
\usepackage[utf8]{inputenc}
\usepackage[english]{babel}

\usepackage{colortbl}
\usepackage{epigraph}
\usepackage{fancyhdr}
\usepackage{graphicx}
\usepackage{hhline}

\usepackage{xcolor}
\definecolor{light-gray}{gray}{0.95}

\title{VT Good Image Protocol \\
a standardization proposal}
\author{Christian Parpart}
\date{2020-08-16 (draft, revision 0)}

\newcommand{\code}[1]{\colorbox{light-gray}{\texttt{#1}}}

\newcommand{\DECRQM}[1]{\code{CSI ? #1 \$ p}}
\newcommand{\DECSET}[1]{\code{CSI ? #1 h}}
\newcommand{\DECRST}[1]{\code{CSI ? #1 l}}

\newcommand{\DA}{\code{DA}}                          % DA

\newcommand\VtModeNum{2027}                          % Grapheme cluster mode Id
\newcommand{\GCON}{\DECSET{\VtModeNum{}}}            % DECSM for enabling grapheme cluster processing
\newcommand{\GCOFF}{\DECRST{\VtModeNum{}}}           % DECRM for disabling grapheme cluster processing
\newcommand{\GCTEST}{\DECRQM{\VtModeNum{}}} % DECRQM for requesting current grapheme cluster processing mode

\newcommand{\GoodImageProtocol}{\code{Good Image Protocol}}

\begin{document}

\maketitle

\tableofcontents

\section{History and current state}

TBD.

\begin{itemize}
    \item DEC REGIS graphics (implemented by xterm)
    \item Sixel graphics (implemented by xterm)
    \item Kitty propriaty (implemented by kitty)
    \item iTerm image protocol (implemented by iTerm, wezterm)
    \item Terminology image protocol (implemented by Terminology)
\end{itemize}

\section{Backwards Compatibility}

Since all other image protocols are pixel based, this image protocol does not retain any
backwards compatibility nor attempts to do that. Instead, the goal is to create
an image protocol that is future proof with todays needs in mind.

\section{Future Compatibility and Stability}

In order to leave room for improvements, the actual VT sequences should be designed in a way
that they allow specifying additional parameters in the future, and that older implementations
can still work with safely ignoring any new parameters.

\section{Requirements}

\begin{itemize}
    \item Headless, multi-headed
    \item Deterministic emulation
    \item Grid cells as the only size unit
    \item Synchronous operations only (no asynchronous operations)
    \item Aspect ratio must be kept by default (support customization though)
    \item Image upload can be decoupled from image display
    \item Cell-based masking on display
    \item Future-reusability of uploaded images for other uses should be possible (such as
        icon-display, desktop-notifications, background images)
\end{itemize}

\section{Feature Detection}

\DA is already used to advertise terminal features, such as sixel, and thus, could be used
to also advertise for the \GoodImageProtocol.

There is some feature detection specification work ongoing, so there may be other ways to detect
\GoodImageProtocol in the future, too.

\section{Performance Considerations}

TBD.

\section{Semantics}

\subsection{Upload Image}

Uploads an image for future render operations.

\begin{itemize}
    \item format; supported: PNG, JPG, BMP
    \item data; image data in the given input format
    \item tag; an uniquily identifiable name in Java-style reverse dot notation, such as
        \code{org.binutils.ls} for the standard \code{ls} executable from the \code{binutils} package.
\end{itemize}

\subsection{Render Image}

Renders an image that has either been previously uploaded already and thus is being referenced
with a unique Id, or with the image data provided inline.

Providing an upload Id along with inline image data is invalid.

\begin{itemize}
    \item Id; optionally referencing a previously uploaded image
    \item x-offset; start render at given x-offset cell of the image
    \item y-offset; start render at given y-offset cell of the image
    \item data; image data if no Id was specified
    \item format; image format if no Id was specified
    \item top; optionally specify top of the rendered target image
    \item left; optionally specify left-side of the rendered target image
    \item width; number of grid cells to render horizontally
    \item height; number of grid cells to render vertically
    \item mask; optional, a boolean stream indicating which grid cells to render to
    \item status: optional, request a success/failure status response from the terminal, by default
        no status result will be sent back to the client application.
\end{itemize}

\subsection{Release Image}

Dereferences the use-cound of a previously uploaded image.

\section{Syntax}

\subsection{Feature Detection}

\subsection{Upload Image}

\subsection{Render Image}

\subsection{Release Image}

\section{References}

\begin{itemize}
    \item https://invisible-island.net/xterm/ctlseqs/ctlseqs.txt
    \item TODO: links to the other image protocols
\end{itemize}

\end{document}
